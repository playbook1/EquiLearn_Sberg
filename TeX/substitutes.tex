% strategic substitutes
\documentclass[a4paper,12pt]{article}  %% important: a4paper first
%
\usepackage[notcite,notref]{showkeys}
\pdfoutput=1
\usepackage{natbib} 
\usepackage{amsthm}
\usepackage{newpxtext,newpxmath} 
\usepackage{microtype}
\linespread{1.10}        % Palatino needs more leading (space between lines)
\usepackage{xcolor}
\usepackage{pict2e} 
\usepackage{bimatrixgame}
\usepackage{tikz} 
\usetikzlibrary{shapes}
\usetikzlibrary{arrows.meta}
\usepackage{amssymb}
%\usepackage{smallsec}
\usepackage{graphicx}
%\usepackage[pdflatex]{hyperref}
\usepackage[hyphens]{url} 
\usepackage[colorlinks,linkcolor=purple,citecolor=blue]{hyperref}
%\usepackage{hyperref}
\urlstyle{sf}
\usepackage[format=hang,justification=justified,labelfont=bf,labelsep=quad]{caption} 
% \input macros-drawtree
\oddsidemargin=.46cm    % A4
\textwidth=15cm
\textheight=23.3cm
\topmargin=-1.3cm
\clubpenalty=10000
\widowpenalty=10000
\predisplaypenalty=1350
\sfcode`E=1000  % normal spacing if E followed by period, as in "EFCE."
\sfcode`P=1000  % normal spacing if P followed by period, as in "NP." 
\newdimen\einr
\einr1.7em
\newdimen\eeinr 
\eeinr 1.7\einr
\def\aabs#1{\par\hangafter=1\hangindent=\eeinr
    \noindent\hbox to\eeinr{\strut\hskip\einr#1\hfill}\ignorespaces}
\def\rmitem#1{\par\hangafter=1\hangindent=\einr
  \noindent\hbox to\einr{\ignorespaces#1\hfill}\ignorespaces} 
\newcommand\bullitem{\rmitem{\raise.17ex\hbox{\kern7pt\scriptsize$\bullet$}}} 
\def\subbull{\vskip-.8\parskip\aabs{\raise.2ex\hbox{\footnotesize$\circ$}}}
\let\sfield\mathcal
\newtheorem{theorem}{Theorem}
\newtheorem{corollary}[theorem]{Corollary}
\newtheorem{example}[theorem]{Example}
\newtheorem{lemma}[theorem]{Lemma}
\newtheorem{proposition}[theorem]{Proposition}
\theoremstyle{definition}
\newtheorem{remark}[theorem]{Remark}
\newtheorem{definition}[theorem]{Definition}
\def\reals{{\mathbb R}} 
\def\eps{\varepsilon}
\def\prob{\hbox{prob}}
\def\sign{\hbox{sign}}
\def\proof{\noindent{\em Proof.\enspace}}
\def\proofof#1{\noindent{\em Proof of #1.\enspace}}
\def\endproof{\hfill\strut\nobreak\hfill\tombstone\par\medbreak}
\def\tombstone{\hbox{\lower.4pt\vbox{\hrule\hbox{\vrule
  \kern7.6pt\vrule height7.6pt}\hrule}\kern.5pt}}
\def\eqalign#1{\,\vcenter{\openup.7ex\mathsurround=0pt
 \ialign{\strut\hfil$\displaystyle{##}$&$\displaystyle{{}##}$\hfil
 \crcr#1\crcr}}\,}
\def\zw#1\par{\vskip2ex{\textbf{#1}}\par\nobreak} 
\newdimen\pix  % bounding box height for eps files
\pix0.08ex
\newsavebox{\figA} 
\parindent0pt
\parskip1.3ex

\title{%
Pricing Game with Demand Inertia Has Strategic Substitutes
}

\author{
}

%\date{Febuary 6, 2012}
\date{\today
\\[1ex]
}

\begin{document}
\maketitle

\begin{abstract}
The two-period pricing game of a ``duopoly with demand
inertia'' seems to have strategic substitutes, not
complements as is usual in pricing games.
This could explain more aggressive behavior.
Multi-period play still has to be analyzed.

% \noindent 
% \textbf{ACM classification:} 
% CCS
% $\to$ Theory of computation
% $\to$ Theory and algorithms for application domains
% $\to$ Algorithmic game theory and mechanism design
% $\to$ Solution concepts in game theory,
% exact and approximate computation of equilibria,
% representations of games and their complexity
% 
% 
% \strut
% 
% \noindent 
% \textbf{AMS subject classification:} 
% 91A18  (Games in extensive form)
% 
% \strut
% 
% \noindent 
% \textbf{JEL classification:} 
% C72 (Noncooperative Games)
% 
% \strut
% 
% \noindent 
% \textbf{Keywords:}
% extensive game,
% correlated equilibrium,
% polynomial-time algorithm,
% computational complexity.
% 
\end{abstract}

\section{Two-stage duopoly pricing game} 
\label{s-two}

We consider the oligopoly game by \citet{Selten1965} for
the special duopoly case of the strategy experiments by
\citet{Keser1992}, first over two periods.

As studied by \citet{Keser1992}, the game is played over
fixed number $T$ of periods, here $T=25$.
Each firm $i$ has a \textit{demand potential} $D_i$
that determines its number of $D_i-p_i$ of sold units of a
good when setting a price $p_i$ (firm $i$'s decision in each
period), with a profit of $p_i-c_i$ per unit for the firm's
production cost $c_i$.
The firms have different costs, $c_1=57$ and $c_2=71$.
The myopic monopoly profit maximizes $(D_i-p_i)(p_i-c_i)$
when $p_i=(D_i+c_i)/2$.

At the start of the $T$ periods, both firms have the same
demand potential $D_1=D-2=200$.
After each period, the cheaper firm gains demand potential
from the more expensive firm in proportion to their price
difference, according to
\begin{equation}
\label{demand}
\begin{array}{rcl}
D_1^{t+1}&=&D_1^t+\frac12({p_{2}^t-p_1^t})\,,
\\[1ex]
D_2^{t+1}&=&D_2^t+\frac12({p_{1}^t-p_2^t})\,.
\end{array}
\end{equation}
The total profits are summed up (there is also a discount
factor of 1 percent per time period that favors early
profits, which we ignore).

We first consider $T=2$ and simplify notation, and in effect
decide what the players should do in the second-to-last
period $T-2$ if we number the periods $t=0,\ldots,T-1$.
This is the last time an interaction takes place, because in
the last round the players choose their myopic optimal
price.

Consider one of the players (low-cost or high-cost) and let
their current demand potential be $d$, the price they are
setting $p$, their cost to be $c$, and the opponent's price
be $p'$.
The current period is $T-2$ and the last period is $T-1$.

In the current period, the player's profit is
$(d-p)(p-c)=(d+c)p-p^2-dc$.
The derivative $\frac d{dp}$ of this is $d+c-2p$, which is
zero for $p=(d+c)/2$ (the myopically optimal price), with
profit $(d-c)^2/4$.
In the last period, the player's demand potential has
changed to $d+\frac{p'-p}2$ according to (\ref{demand}).
Consequently, their last-period optimal profit will be
$(d-c+\frac{p'-p}2)^2/4$ or
\[
(d-c)^2/4+(d-c){(p'-p)}/4+(p'^2-2p'p+p^2)/16\,.
\]
The derivative $\frac d{dp}$ of the sum over of these two
terms over the two periods is
\begin{equation}
\label{twop}
d+c-2p-\frac{d-c}4-\frac{p'}8+\frac{p}8
~=~ 
\frac{3d}4+ \frac{5c}4-\frac{p'}8-\frac{15p}8
\end{equation}
which when set to zero gives
\begin{equation}
\label{0}
p=\frac{2d}5+\frac{2c}3-\frac{p'}{15}\,.
\end{equation}
As a sanity check, if both players choose the same price
($p=p'$, which is normally not the case) in (\ref{twop}),
then $p=\frac38d+\frac58c$, which means that rather than
choosing the monopoly price $\frac12d+\frac12c$, the player
should move $\frac14$ of the distance from $\frac{d+c}2$ to
$c$ towards the cost $c$, which is quite a bit of
``shading'' from the monopoly price.
For example, if $d=207$ and $c=57$ (the considered player
has low cost), then $p=113.25$, which is $18.75$ below the
monopoly price of $132$.
If $d=193$ and $c=71$ (the considered player
has high cost), then $p=116.75$, which is $15.25$ below the
monopoly price of $132$ (and the two prices are not the same
with these demand potentials).

The crucial observation in (\ref{0}) is that a
\textit{lower} (more aggressive) opponent price $p'$
induces a \textit{higher} (less aggressive) best-response
price $p$ and vice versa (although the dependence is not
strong), which defines the two price choices as
\textit{strategic substitutes} (rather than complements).
This is therefore akin to the standard linear
\textit{Cournot} model of quantity competition.
In this scenario, both players would prefer to be
leader (first mover) over simultaneous action over being
follower (second mover).




% \section{Strategic substitutes}
% 
% Pricing games are usually games of \textit{strategic
% complements}, that is, the best response to more aggressive
% behavior to be more aggressive.
% That is, a low price of the opponent induces a low price as
% a best response, and high price induces a high rice as a
% best response.
% 
% However, the present game has a delayed interaction effect,
% which seems to correspond to \textit{strategic
% substitutes} where the best response \textit{decreases}
% rather increases with the opponent's price.
% Namely, if the opponent chooses a lower price, then the
% player's own demand potential decreases in the next period.
% If that demand potential is $y$ and own cost is $c$, then
% the myopic best response is $(y+c)/2$, which is increasing
% and decreasing with $y$.
% But a lower $y$ resulted from a lower opponent price ...
% which suggests complements again.

% \cite{vS2010}

\small
\bibliographystyle{book}
\bibliography{bib-evol} 

\end{document}

